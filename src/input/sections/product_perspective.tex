\section{Product Perspective}
BlueNRG is a network processor. This means it implement in it, the stack of both the host and the controller. But for the intent of this project, we are interested only in the controller stack of BlueNRG as the host part of the stack will be provided by a Linux host. As is defined in the specification a host and a controller must communicate using the HCI protocol. And Bluez is of course capable of providing this interface to our controller: BlueNRG. Since BlueNRG uses SPI as the transport layer, there is a need to write a SPI driver that can take HCI commands from Bluez stack and send them over to BlueNRG. The response to HCI commands is of course HCI events. These are communicated using the same channels i.e. SPI. \\
Once this host and controller setup is ready, it is time to communicate with another Bluetooth Low Energy complaint device. The communication between this other device and our host (the Linux machine) will happen through one of many Bluetooth Low Energy profiles in the application layer. The driver and BlueNRG need not concern itself with what these applications are. All it needs to care about is the HCI commands that is going to receive from the host over SPI and sending back appropriate responses for the send commands. This is possible because any operation that is needed to performed by the application will eventually be converted into a HCI command. All of this is taken care of by the Bluez stack between its user-space and then the kernel side components discussed earlier. \\
The follow of control or data is summarized in the following flow diagram.
