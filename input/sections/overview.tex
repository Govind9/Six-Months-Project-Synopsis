\section{Overview}
In order to have two devices communicate with each other using Bluetooth, one needs to ensure that both devices are equipped with the enough hardware capability and apt software support. The fundamental component of a Bluetooth device is the Bluetooth radio and some necessary subset of the Bluetooth stack. BlueNRG is one such device. Like any other Bluetooth hardware device, BlueNRG can be used as an adapter to work with a host system that otherwise does not have the hardware to function Bluetooth. This use case makes BlueNRG a peripheral to a host i.e. A system on chip. \\
The host system must be able to communicate with BlueNRG in order set straight the relationships and interaction between various portions of the stack. These communications are of course handled by the Operating System. Furthermore for the OS to communicate with an external peripheral, device drivers are employed. And this is the whole idea of this project: creating a device driver for BlueNRG so that it can be used in conjunction with systems running on Linux based operating systems.\\
The device driver ensures that the peripheral is able to communicate with the host but the way this communication is to be conducted depends on what physical connection options are offered by the peripheral. In case of BlueNRG, the physical interface is SPI. Once the means of communication are fixed, there comes the question of deciding which part of the protocol is to reside on which side, the host or the peripheral. It will be seen in later chapters as to how the whole stack has been fragmented in this project.\\
Linux is written in C and so are the device drivers for Linux.
