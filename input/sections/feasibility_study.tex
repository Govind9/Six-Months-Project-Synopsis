\section{Feasibility Study}
As stated in the requirement analysis section, there are two things that are needed to considered here:
\subsection{ Does BlueNRG support what Bluez offers?}
The answer is yes. Bluez interfaces with any device or adapter using the Host Controller Interface which is essentially a protocol standard defined in the Bluetooth specification itself as a means to communicate between the host and the controller. Since BlueNRG is responsive to HCI commands, all seems to be set here. The communication between the host and the controller are to be had through HCI commands (send from the host) and HCI events (generated by the controller in response to the commands send by the host). According to the ACI programing manual, HCI is indeed the subset of ACI. So BlueNRG should be able to respond to HCI commands offered by Bluez of Linux.
\subsection{Does Linux supports what BlueNRG uses to communicate through?}
BlueNRG is based on SPI. Any form of communication between the host and BlueNRG is to be had through the Serial Peripheral Interface. While there isn’t a SPI based Bluetooth driver in Linux at the time of this writing, but there certainly is a rich SPI support in Linux. There are many network drivers that allow the communication between the host and the controller through SPI. So following the same footsteps it should be possible to send commands (HCI) from the host to the controller using SPI.\\
Considering the above things, it seems feasible to write a driver for BlueNRG in Linux. The driver shall essentially act as a bridge between Bluez protocol stack residing partly in the user space and partly in the kernel space of Linux and BlueNRG. Every Bluetooth action that is needed to be performed will be in the form of an HCI command, which is what the driver should expect from Bluez and interface it through SPI to BlueNRG.
