\section{Requirement Analysis}
There are many Bluetooth adapters out there that have well documented and tested support in Linux based operating systems. Linux being an open source kernel is always inviting people to look into its internal working and stack flow in order to contribute in it.\\
STMicroelectronics excels at grabbing newer opportunities and always be ready with solutions that are on par or even transcends other solutions in the market in a wide range of technologies. The Bluetooth technology is no different. ST has its fair share of Bluetooth modules. Along with the Bluetooth Smart technology, ST has also offered modules that serve complaint to that specification. But thing to be considered here is that even though ST has a wide range of Bluetooth solutions, none has Linux support. \\
So the major requirement of this project was to take a Bluetooth module by ST and add its support to Linux. And again Linux being open source makes the job much simpler. The module that was considered is the BlueNRG module which is Bluetooth 4.0 complaint. This means that it has the much coveted Bluetooth Smart (Low Energy) features. So the whole requirement of this project can be summed up as adding support (writing a device driver) in Linux (because it is open source and one of the most popular OS kernels in the world) for the BTLE module offered by STMicroelectronic i.e. BlueNRG.\\
In order to fulfill the above stated requirement it is essential to look into both sides of the coin: Linux driver development and how is the Bluetooth stack being represented there; and then BlueNRG and what portion of the stack is on it and how much of it is needed in order to work with Linux.\\
The official Bluetooth protocol stack for Linux is Bluez. Bluez is designed in such a way that it offers the HCI interface to any adapter that wish to communicate through it. So for a Bluetooth adapter to be connected to a Linux host, and wanting to utilize the Bluez protocol stack, it needs to extend the HCI interface. And since HCI is also part of the Bluetooth standard, it is not very hard have all Bluetooth devices follow the same protocol. If communication between the host and the controller is to be had using the Bluez protocol stack then it has to happen through the Host Controller Interface as defined in the Bluetooth specification document.\\
Now comes the question of whether BlueNRG is capable of HCI communication, to which the answer is a definite yes. As it will be seen later in this report, BlueNRG is designed to communicate using the Application Controller Interface (ACI), which is a proprietary protocol. But the good news is that ACI is almost a super set of the standard HCI protocol. Since ACI is proprietary and Linux Bluetooth stack offers communication to all devices through only the HCI protocol, it becomes essential to use only the HCI subset available for BlueNRG. Another important point to consider is that how the connection is to be made between the Host (running on Linux) and BlueNRG controller. BlueNRG is based on Serial Peripheral Interface, so that is exactly what is required to be used for the connections. 
